\documentclass[12pt]{article}

\usepackage[utf8]{inputenc}
\usepackage[english]{babel} 
\usepackage{graphicx}
\usepackage{hyperref}
\usepackage{geometry}
\usepackage{titlesec}
\usepackage{csquotes}
\usepackage{biblatex}
\usepackage{array}
\titlelabel{\thetitle.\quad}
\geometry{a4paper, margin=2.5cm}

\title{Portfolio Credit Risk models}
\author{Anamarija Potokar, Živa Artnak}
\date{}

\begin{document}

\maketitle

\section{Introduction}

Credit risk is the possibility that a borrower will not meet its contractual obligations, leading to a loss for the lender. In practice, this risk is usually described in terms of probability of default and the loss given default. There are two broad ways in which this risk is modelled. In default-mode approaches, only the event of default matters, meaning a borrower either survives the risk horizon or defaults, and losses occur only in the latter case. In rating-migration or mark-to-market approaches, the focus is broader. Changes in a borrower's credit quality, reflected by transitions between rating categories, generate gains and losses through spread changes even if no default occurs. Both perspectives are important, but they lead to different modelling choices and risk measures. \\

\noindent Traditionally, banks assessed credit risk mainly at the level of individual transactions, often with a simple classification of loans into "good" and "bad"; this view is no longer sufficient. Experience has shown that virtually any exposure can become "bad" under adverse economic conditions, and that the risk of a bank cannot be understood by looking at single loans in isolation. What matters is the joint behaviour of many borrowers and the way losses can accumulate in stressed states of the economy. As a result, financial institutions increasingly measure and manage credit risk at the portfolio level, asking not only what is the risk of this loan, but also how does this loan affect the risk of the whole book. \\

\noindent Several developments have reinforced this portfolio perspective. Competitive pressure and narrow lending margins leave little room for error in selecting and pricing individual exposures; small mistakes can have a large impact on profitablity once they are aggregated. At the same time, markets for syndicated loans, securitisations and credit derivatives have become more liquid, giving banks more tools to actively reshape their credit portfolios after origination. Diversification across borrowers, sectors and regions, and the timing of adjustments to the portfolio, increasingly determine whether a bank earns a profit or suffers a loss in a given period. \\

\noindent To take advantage of these opportunities, however, banks must be able to answer a number of technical questions. They need to quantify the distribution of portfolio loss over a given horizon, not only its expected value but also extreme outcomes that occur with low probability. They must understand how different macroeconomic scenarios or sector-specific shocks affect this loss distribution, how changes in the portfolio mix alter concentration and diversification, and how risk-based pricing and capital allocation should reflect expected losses and the cost of holding credit risk capital. \\

\noindent This essay addresses these questions by reviewing and comparing the main portfolio credit risk models used in risk management and regulation. We discuss bottom-up versus top-down approaches, default-mode versus mark-to-market frameworks, and structural versus reduced-form models, and we relate them to the regulatory capital model used in the Basel II/III Internal Ratings-Basel approach. The aim is not to provide a full mathematical treatment of each model, but to explain their assumptions, inputs and outputs, to highlight their strengths and weaknesses, and to draw clear conclusions about their suitability for different risk management purposes. \\

\section{Top-down vs. Bottom-Up models}

\noindent Portfolio credit risk models can be broadly divided into bottom-up and top-down approaches, depending on whether credit
risk is modeled at the level of individual borrowers or directly at the level of portfolio segments. This distinction reflects 
two fundamentally different ways of thinking about credit risk aggregation, modeling losses as the outcome of individual 
firm-level default mechanisms, or modeling them as the result of common systematic risk factors affecting groups of exposures simultaneously. \\

\noindent Bottom-up models build portfolio risk starting from the individual instrument or borrower. Each exposure is characterized by 
borrower-specific credit risk parameters such as the probability of default, loss given default, and exposure at default, while 
dependence across borrowers is introduced through correlated risk drivers. In this framework, portfolio losses arise from the joint 
realization of individual default events. This approach is closely related to models that 
provide an explicit economic or statistical description of default at the firm level, including structural and intensity-based 
specifications. Aggregation across borrowers then delivers the portfolio loss distribution. \\

\noindent A key advantage of bottom-up models is their high degree of granularity. They allow risk managers to measure concentration 
risk, assess marginal risk contributions of individual exposures, and perform detailed scenario analyses at the borrower level. 
This makes them particularly suitable for corporate loan portfolios and portfolios of traded credit instruments, where firm-level 
information is available and economically meaningful. However, the bottom-up approach is also data-intensive and requires strong 
assumptions regarding default dependence, which must be calibrated using limited historical default data or proxy measures such 
as asset correlations. \\

\noindent In contrast, top-down models describe credit risk directly at the level of portfolio segments, such as industries, 
regions, or rating classes. Instead of modeling each borrower separately, these models focus on common sources of risk 
(most notably macroeconomic conditions) that drive default rates and losses across groups of borrowers. From this perspective, 
idiosyncratic risk is largely diversified away, and portfolio losses are primarily determined by the evolution of systematic 
factors. Such models are particularly useful when individual default mechanisms are difficult to 
observe or estimate reliably, but aggregate default behavior exhibits a stable relationship with economic conditions. \\

\noindent Top-down models are therefore especially well suited for stress testing and scenario analysis, where the objective 
is to assess the sensitivity of portfolio losses to adverse macroeconomic developments. They are also computationally simpler and 
less demanding in terms of input data, which makes them attractive for large, relatively homogeneous portfolios, such as retail credit 
exposures. Their main limitation lies in the lack of borrower-level detail, which prevents a precise assessment of individual risk 
contributions and limits their usefulness for pricing and active portfolio optimization. \\

\noindent In practice, the distinction between top-down and bottom-up models highlights a fundamental trade-off between economic detail
and tractability. Bottom-up models provide richer insights into individual credit risk and portfolio composition, while top-down models 
offer a clearer link between credit losses and systematic risk factors. For this reason, modern risk management frameworks often combine 
both approaches, applying them selectively depending on the portfolio structure and the specific risk management objective. \\

\section{Default-mode vs. Mark-to-market models}

\noindent A second fundamental distinction among portfolio credit risk models concerns whether losses are modeled only through default events or 
whether changes in credit quality short of default are also taken into account. This leads to a classification into default-mode and mark-to-market models, 
a distinction emphasized in both academic literature and practical risk management frameworks. \\

\noindent Default-mode models focus exclusively on the occurrence of default over a given risk horizon. In these models, each exposure either survives or defaults. 
Changes in credit quality that do not lead to default are irrelevant for the loss calculation. As a result, default-mode models are primarily concerned with 
estimating the distribution of default losses, rather than the market value of the portfolio. CreditRisk+ is a prominent example of this approach, as it models 
defaults as random events driven by default intensities and derives the portfolio loss distribution using actuarial techniques. \\

\noindent The main strength of default-mode models lies in their conceptual simplicity and computational efficiency. By abstracting from market value fluctuations, 
these models avoid the need to specify credit spreads, rating transition dynamics, or valuation models for credit instruments. This makes them particularly attractive 
for applications such as economic capital calculation, where the primary objective is to quantify extreme losses at high confidence levels. From a theoretical perspective, 
default-mode models are consistent with the view that credit risk materializes primarily through rare but severe events.
However, by construction, default-mode models ignore potentially important sources of risk. In practice, the market value of credit portfolios can fluctuate significantly 
even in the absence of defaults, due to changes in credit spreads, rating migrations, and shifts in investors’ risk perceptions. These effects are particularly relevant 
for portfolios containing traded credit instruments or loans that are marked to market. \\

\noindent Mark-to-market models explicitly account for such changes in credit quality. In addition to default losses, they capture gains and losses arising from rating 
migrations and spread movements, even when no default occurs. In this framework, portfolio credit risk is measured as the distribution of changes in portfolio value over 
the risk horizon. CreditMetrics, developed by J.P. Morgan, is the canonical example of a mark-to-market model. It models rating transitions using empirical transition 
matrices and revalues each exposure under different rating scenarios, including default, thereby producing a full distribution of portfolio value changes. \\

\noindent Mark-to-market models offer a more comprehensive view of credit risk, particularly for actively managed portfolios and portfolios of traded instruments. 
They are well suited for short to medium term risk horizons and for risk measurement frameworks that rely on value-at-risk or similar metrics. At the same time, they require more extensive modeling assumptions, including assumptions about rating transition probabilities, recovery rates, and the valuation of credit instruments under different credit states. \\

\noindent In practice, the choice between default-mode and mark-to-market models depends on the risk management objective and the nature of the portfolio. Default-mode models are typically preferred for long-horizon capital adequacy and regulatory purposes, while mark-to-market models are more appropriate for portfolios where interim valuation changes are economically significant. Many institutions therefore use both approaches in parallel, recognizing that they capture complementary dimensions of portfolio credit risk.

\section{Conditional vs. Unconditional models}

\noindent Another important distinction in portfolio credit risk modeling concerns whether default probabilities (and, more generally, losses) are treated as fixed, long-run quantities or as outcomes that vary with the state of the economy. This leads to the classification into unconditional and conditional models. Although the difference can sound technical, it matters a lot in practice because it determines whether a model produces a relatively stable, ``average'' view of risk or whether it explicitly reacts to changing macro-financial conditions and can be used for scenario-based analysis. \\

\noindent In an unconditional model, key credit risk inputs such as the probability of default and loss given default are typically treated as fixed over the risk horizon. The model aims to produce an overall loss distribution that reflects a long-run average across many possible economic states. Institutions often implement this perspective using so-called through-the-cycle estimates, which are constructed to be relatively stable over time and not overly sensitive to short-run fluctuations in the economy. A practical advantage of this approach is that it supports capital planning and performance measurement in a way that does not change dramatically from quarter to quarter. It also reduces the danger of procyclical decision-making, where risk measures rise sharply in downturns and fall sharply in booms, amplifying the cycle. \\

\noindent Importantly, unconditional models do not ignore dependence across borrowers. They still incorporate the idea that borrowers are exposed to common drivers and that defaults can cluster. The key point is that the dependence structure and default probabilities are typically parameterized in a time-invariant way rather than explicitly linked to observable economic conditions. The model therefore captures average dependence, but it does not directly explain how default risk changes when the macro environment deteriorates. \\

\noindent In contrast, conditional models treat credit risk as state-dependent. They explicitly represent the idea that default rates and losses rise in recessions and fall in expansions. The basic modeling logic is that borrowers are exposed to common systematic conditions (such as broad macroeconomic factors, sector-specific conditions, or financial market stress), and that these conditions influence many borrowers at the same time. Conditional models are therefore especially useful when the goal is not only to measure risk, but to understand why risk changes and how the portfolio might behave under adverse scenarios. \\

\noindent A central concept in many conditional frameworks is conditional independence: borrowers may be treated as largely independent once the systematic environment is fixed, but they become dependent in the unconditional sense because they all respond to the same underlying conditions. This is not just a convenient assumption; it reflects the empirical observation that default clustering is largely driven by shared economic stress. As a result, conditional models provide a natural foundation for stress testing. Rather than simply reporting a single number for risk, they allow a risk manager to translate a macroeconomic scenario into implied probabilities of default, losses given default, default rates, and ultimately portfolio losses. \\

\noindent The conditional versus unconditional distinction also appears in how institutions discuss estimation of probability of default. A common practical split is between point-in-time probabilities of default, which reflect current conditions and tend to respond quickly to changes in the economy, and through-the-cycle probabilities of default, which are smoother and closer to long-run averages. Conditional models align naturally with point-in-time estimation and monitoring, while unconditional models align more closely with through-the-cycle estimates used for stable capital frameworks. Many institutions use both in parallel, since they serve different objectives: point-in-time measures are useful for timely risk signals and pricing, while through-the-cycle measures support longer-term capital and planning.

\section{Structural vs. Reduced-form models}

\noindent A further key distinction concerns the way default is modeled at the level of a single borrower. The two dominant paradigms are structural (firm-value) models and reduced-form (intensity-based) models. Both can be used within bottom-up portfolio frameworks, and both can be combined with systematic factors to generate realistic portfolio dependence. However, they differ fundamentally in their interpretation of what default is and in how they connect credit risk to economic or market data. \\

\noindent Structural models explain default as the outcome of a borrower’s underlying economic condition. In these models, default occurs when the value of the borrower’s assets becomes insufficient relative to its liabilities, so default is tied to the borrower’s balance sheet and asset dynamics. The classical intuition is that equity resembles a residual claim on the firm’s assets, which provides an economically meaningful link between leverage, volatility, and default risk. Structural models are attractive because default is not introduced as an arbitrary event; instead, it arises from an explicit mechanism that reflects firm fundamentals. This makes the framework particularly suitable for corporate credit, where concepts such as firm value, leverage, and debt obligations are central. \\

\noindent At the same time, structural models can be difficult to implement in practice. A borrower’s total asset value is typically not directly observable, and real-world defaults often occur due to liquidity pressures, covenant breaches, or strategic behavior rather than a clean, gradual deterioration of an underlying ``asset value.'' Moreover, observed credit spreads often incorporate premia for liquidity and risk aversion that structural models do not automatically capture. As a result, practical structural implementations often require additional assumptions and approximations. \\

\noindent Reduced-form models take a different route: instead of deriving default from firm value crossing a threshold, they treat default as a stochastic event whose likelihood is described by a default intensity (or hazard rate). The intensity can depend on observable covariates and can be calibrated to match market prices of credit instruments such as corporate bonds and credit default swaps. This market-consistent calibration is a major advantage: reduced-form models are often well suited for pricing and risk management of traded credit products, where the model must connect closely to observed spreads. In addition, these models are naturally compatible with the idea that defaults can occur unexpectedly from the market’s perspective, capturing sudden credit events more directly than many structural frameworks. \\

\noindent The main limitation of reduced-form models is interpretability. Because default is not derived from an underlying balance-sheet mechanism, the intensity is a modeling input rather than an economically derived outcome. This can make it harder to link default risk to firm fundamentals in a transparent way. Furthermore, calibrating intensities to market spreads requires conventions about recovery and a recognition that spreads reflect not only expected default losses but also risk premia and other market factors. \\

\noindent In practice, the choice between structural and reduced-form modeling depends on the use case and the data. Structural models are often preferred when economic interpretation and a connection to borrower fundamentals are central, while reduced-form models are widely used when the objective is valuation and market calibration. Many modern frameworks combine elements of both, aiming to preserve economic intuition while remaining flexible enough to fit market data.

\section{Modeling dependence and default clustering}

\noindent Regardless of whether a model is conditional or unconditional, and regardless of whether default is modeled structurally or via intensities, the core portfolio challenge remains the modeling of default dependence. The tail of the loss distribution is driven by the possibility that many borrowers default in the same period, particularly under adverse economic conditions. This feature distinguishes credit portfolios from many other risk types: diversification can reduce idiosyncratic risk, but systematic downturn risk remains and dominates extreme outcomes. \\

\noindent The most common practical approach is to use factor structures, where each borrower’s creditworthiness is influenced by one or more systematic factors and an idiosyncratic component. This approach is appealing because it provides an intuitive explanation for default clustering while remaining tractable and scalable to large portfolios. Another approach is to use copula models, which specify dependence more flexibly and can, depending on the choice, represent stronger co-movement in extreme states. However, copula-based results can be sensitive to specification choices, which increases model risk and raises calibration challenges. More elaborate approaches attempt to represent \textbf{contagion} effects, where one default raises the risk of others through economic linkages, but these models often require strong assumptions and detailed data that are not always available. \\

\noindent Because dependence assumptions have a particularly strong effect on extreme quantiles, institutions often treat them as a main source of model risk, and complement model outputs with sensitivity analysis and conservative overlays.

\section{Regulatory perspective: Basel II/III and the IRB capital model}

\noindent Portfolio credit risk modeling is not only a tool for internal risk management but also one of the building blocks of modern banking regulation. 
The Basel framework is motivated by a simple concern: banks must be able to absorb credit losses in adverse states of the world while continuing to function. 
Because credit losses are typically modest in normal times but can rise sharply and simultaneously in downturns, regulation has to focus on portfolio-wide outcomes 
and on the tail of the loss distribution rather than on the risk of a single loan in isolation. \\

\noindent Basel II formalized this portfolio perspective through the Internal Ratings-Based approach. Under Internal Ratings-Based, regulatory capital is linked to a 
small set of key credit risk parameters: the probability of default, loss given default, exposure at default, and maturity. The idea is to allow bank-specific 
risk assessments to matter, but within a standardized and conservative portfolio framework that produces comparable capital requirements across institutions. 
A central feature of this framework is that default dependence is driven by a common systematic environment, which captures the fact that defaults cluster when 
macroeconomic conditions deteriorate. \\

\noindent Within this setup, the Basel logic distinguishes between expected loss and unexpected loss. Expected loss represents the average credit 
loss a bank anticipates over the horizon, given the characteristics of its borrowers and historical experience. In principle, expected loss should be covered through 
pricing (interest margins), loan loss provisions, and reserves. Unexpected loss refers to losses that arise in worse-than-average conditions and reflect both uncertainty 
and the joint nature of defaults in stress. Regulatory capital is designed primarily to absorb these unexpected losses and thereby protect depositors and the stability of 
the financial system. \\

\noindent This distinction is closely connected to provisioning and accounting. Modern approaches to provisioning emphasize earlier recognition of credit deterioration and 
forward-looking loss estimates. In practice, provisioning and capital interact: increasing provisions reduces profits and equity in the short run, but it can also reduce the 
portion of realized credit losses that would otherwise have to be absorbed by capital. This interaction becomes especially important in downturns, when default rates rise and 
banks simultaneously face pressure to increase provisions. \\

\noindent Basel III retained the core Internal Ratings-Based logic but strengthened the overall regime after the global financial crisis. The reforms did not reject model-based capital, but they 
reduced reliance on any single model output by tightening the definition of eligible capital, increasing required buffers, and adding constraints that sit alongside risk-weighted 
requirements. The broader lesson is that even a well-designed portfolio model can be undermined by optimistic inputs, incentives to minimize risk weights, or structural changes in 
credit markets. Additional buffers and constraints are therefore intended to improve resilience when losses materialize in conjunction with funding stress and wider market disruptions. \\

\noindent Despite its benefits, the regulatory framework also has well-known limitations. The stylized assumptions that support tractability and comparability are less appropriate 
for concentrated corporate portfolios, where single-name exposures remain material. Likewise, a single systematic driver may be too simple to capture sectoral concentration, geographic 
differences, and multi-dimensional macroeconomic shocks. Another challenge is calibration: regulatory approaches necessarily rely on parameter estimates in a rare-event setting, and 
default and recovery data are limited precisely in the part of the distribution that matters most for solvency. \\

\noindent A final concern is cyclicality. If key inputs become too sensitive to current conditions, capital requirements can rise sharply in downturns, when banks are least able 
to raise new capital and when lending is most valuable for the real economy. The buffer system and supervisory judgment are partly designed to address this tension, but it remains a central 
theme in both regulation and internal practice. Overall, the Internal Ratings-Based framework is best viewed as a robust and standardized baseline for solvency, rather than as a complete description of a bank’s 
credit risk under all conditions.

\section{Practical considerations and model risk}

\noindent Implementing portfolio credit risk models in practice requires far more than selecting a model class. The results depend critically on data quality, parameter 
estimation choices, portfolio segmentation decisions, and governance. In real institutions, the modeling process is therefore inseparable from questions of 
measurement, validation, and model risk management. \\

\noindent Data and parameter estimation are the first challenge. Probabilities of default must often be estimated from limited historical default experience, 
especially for high-grade portfolios where defaults are rare. This makes estimates noisy and sensitive to the chosen time window, the treatment of missing information, 
and the use of external data sources. Loss given default is also difficult: recoveries depend on seniority, collateral, legal environment, and the economic cycle. 
Empirically, recoveries tend to be lower in downturns, which means that assuming a constant loss given default can understate tail risk. Exposure at default introduces another source 
of uncertainty, particularly for revolving facilities and credit lines, where utilization can rise exactly when borrower quality deteriorates. \\

\noindent A second challenge is portfolio mapping and segmentation. Many models require grouping exposures by rating, industry, geography, or product type. 
These choices affect dependence and concentration estimates. A segmentation that is too coarse can hide concentrations, while segmentation that is too fine can create 
unstable parameter estimates. This is one reason why large institutions frequently maintain multiple portfolio views, such as a regulatory segmentation, a business 
segmentation for management, and a stress testing segmentation aligned with macro drivers. \\

\noindent Dependence modeling is often the dominant driver of extreme loss estimates. Small changes in asset correlation assumptions, factor loadings, or tail 
dependence choices can materially change high-quantile risk measures. Because default clustering is most visible in crises, historical data may not provide enough 
information to precisely calibrate tail behavior. For this reason, dependence parameters are often treated conservatively and are supplemented with expert judgment, 
benchmarking, and scenario-based overlays. \\

\noindent These issues motivate a strong emphasis on model validation. Validation typically includes checks of conceptual soundness, benchmarking against 
alternative models, sensitivity analysis, and outcome-based reviews where feasible. For credit risk, outcome validation is challenging because true tail events occur 
rarely, so backtesting has limited statistical power. As a result, validation often focuses on stability, plausibility under stress, and consistency with observed 
credit dynamics rather than narrow statistical backtests alone. \\

\noindent Finally, institutions must manage model risk and governance. Portfolio credit risk models influence pricing, limits, capital allocation, and 
strategic portfolio decisions, so errors can have significant consequences. Model governance therefore typically includes documentation of assumptions, independent review, 
controlled change processes, and clear communication of limitations. Good practice also requires that senior management understands that model outputs are not facts, 
but conditional estimates that reflect assumptions and uncertainty. \\

\noindent In this sense, robust credit risk management combines models with stress testing, sensitivity analysis, and judgment. Models are essential for consistency and 
scale, but they are most valuable when they support decision-making, highlight concentrations, and provide a transparent framework for discussing credit risk under 
normal and adverse conditions.

\section{Prominent portfolio credit risk frameworks in practice}

\noindent The classifications discussed above are not only theoretical; they correspond to a set of well-known frameworks that shaped industry practice. 
A short comparison helps clarify how different modeling choices translate into different use cases. \\

\noindent CreditMetrics is the canonical mark-to-market framework. Its central feature is the explicit modeling of rating migrations and the 
revaluation of exposures under different credit quality states. As a result, CreditMetrics is most naturally aligned with portfolios of traded credit instruments 
and risk measurement over horizons where changes in spreads and rating quality are economically meaningful. The model is particularly useful for value-at-risk style 
risk reporting and for understanding how the distribution of portfolio value changes is affected by credit quality transitions. \\

\noindent CreditRisk+ is a canonical default-mode approach. It abstracts from spread changes and focuses on default counts and default losses. 
Its actuarial perspective and computational efficiency made it attractive for economic capital calculations and for large portfolios where the primary concern is the 
distribution of default losses rather than interim mark-to-market fluctuations. Because it is default-focused, it is often discussed in contexts where defaults are 
the dominant economic loss mechanism and where valuation changes are either not observed or not the main management objective. \\

\noindent CreditPortfolioView represents a conditional, macro-linked perspective. Instead of treating default probabilities as fixed, it emphasizes 
that default and migration behavior changes systematically with macroeconomic conditions. This makes it especially useful for stress testing and scenario analysis, 
where the goal is to translate a macro scenario into portfolio-level loss implications. The framework highlights the importance of linking portfolio risk to economic 
drivers that can be monitored and communicated. \\

\noindent Finally, the Basel Internal Ratings-Based framework can be seen as a standardized regulatory application of a conditional factor approach, designed to be 
consistent and scalable across institutions. Its purpose is not to be the most detailed model, but to provide a conservative and transparent baseline for determining 
capital requirements. In practice, institutions often manage their portfolios using a richer internal framework while still mapping results to the regulatory structure 
for reporting and compliance. \\

\noindent Taken together, these frameworks illustrate that portfolio credit risk modeling is not about a single ``best'' model. Instead, different models exist because 
institutions face different objectives: valuation and risk reporting, default loss capital estimation, stress testing, regulatory compliance, and portfolio optimization.

\section{Conclusion}

\noindent Portfolio credit risk models provide a structured way to translate borrower-level credit risk into portfolio-level loss distributions and risk measures. The most important modeling choices reflect trade-offs rather than purely technical preferences. Top-down and bottom-up approaches balance tractability against borrower-level detail; default-mode and mark-to-market models differ in whether they focus only on default losses or also on valuation changes due to migration and spread movements; unconditional and conditional models differ in whether they aim for a stable long-run view of risk or an explicitly state-dependent, scenario-sensitive view; and structural and reduced-form models differ in whether they explain default through borrower fundamentals or treat it as a stochastic event calibrated to market data. \\

\noindent In practice, institutions often use multiple models in parallel because these approaches capture complementary dimensions of portfolio credit risk. The regulatory framework provides an influential benchmark, but effective risk management also requires recognizing uncertainty, testing sensitivity to assumptions, and understanding how portfolio composition and systematic conditions interact to shape extreme loss outcomes.

\end{document}
