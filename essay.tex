\documentclass[12pt]{article}

\usepackage[utf8]{inputenc}
\usepackage[english]{babel} 
\usepackage{graphicx}
\usepackage{hyperref}
\usepackage{geometry}
\usepackage{titlesec}
\usepackage{csquotes}
\usepackage{biblatex}
\usepackage{array}
\titlelabel{\thetitle.\quad}
\geometry{a4paper, margin=2.5cm}

\title{Portfolio Credit Risk models}
\author{Anamarija Potokar, Živa Artnak}
\date{}

\begin{document}

\maketitle

\section{Introduction}

Credit risk is the possibility that a borrower will not meet its contractual obligations, leading to a loss for the lender. In practice, this risk is usually described in terms of \textbf{probability of default} and the \textbf{loss given default}. There are two broad ways in which this risk is modelled. In \textbf{default-mode} approaches, only the event of default matters, meaning a borrower either survives the risk horizon or defaults, and losses occur only in the latter case. In \textbf{rating-migration} or \textbf{mark-to-market} approaches, the focus is broader. Changes in a borrower's credit quality, reflected by transitions between rating categories, generate gains and losses through spread changes even if no default occurs. Both perspectives are important, but they lead to different modelling choices and risk measures. \\

\noindent Traditionally, banks assessed credit risk mainly at the level of individual transactions, often with a simple classification of loans into "good" and "bad"; this view is no longer sufficient. Experience has shown that virtually any exposure can become "bad" under adverse economic conditions, and that the risk of a bank cannot be understood by looking at single loans in isolation. What matters is the joint behaviour of many borrowers and the way losses can accumulate in stressed states of the economy. As a result, financial institutions increasingly measure and manage credit risk at the portfolio level, asking not only what is the risk of this loan, but also how does this loan affect the risk of the whole book. \\

\noindent Several developments have reinforced this portfolio perspective. Competitive pressure and narrow lending margins leave little room for error in selecting and pricing individual exposures; small mistakes can have a large impact on profitablity once they are aggregated. At the same time, markets for syndicated loans, securitisations and credit derivatives have become more liquid, giving banks more tools to actively reshape their credit portfolios after origination. Diversification across borrowers, sectors and regions, and the timing of adjustments to the portfolio, increasingly determine whether a bank earns a profit or suffers a loss in a given period. \\

\noindent To take advantage of these opportunities, however, banks must be able to answer a number of technical questions. They need to quantify the distribution of portfolio loss over a given horizon, not only its expected value but also extreme outcomes that occur with low probability. They must understand how different macroeconomic scenarios or sector-specific shocks affect this loss distribution, how changes in the portfolio mix alter concentration and diversification, and how risk-based pricing and capital allocation should reflect expected losses and the cost of holding credit risk capital. \\

\noindent This essay addresses these questions by reviewing and comparing the main portfolio credit risk models used in risk management and regulation. We discuss bottom-up versus top-down approaches, default-mode versus mark-to-market frameworks, and structural versus reduced-form models, and we relate them to the regulatory capital model used in the Basel II/III Internal Ratings-Basel approach. The aim is not to provide a full mathematical treatment of each model, but to explain their assumptions, inputs and outputs, to highlight their strengths and weaknesses, and to draw clear conclusions about their suitability for different risk management purposes. \\

\noindent Building on these considerations, this report provides a structured overview of the main portfolio credit risk 
models used in risk management and regulation. Following the framework presented in the Risk Management lectures, the models 
are classified along several key dimensions: bottom-up vs. top-down approaches, default-mode vs. mark-to-market models, 
conditional vs. unconditional models of default probability, and structural vs. reduced-form models. The aim of the report 
is to explain the underlying assumptions, inputs, and outputs of these approaches, to highlight their respective strengths 
and limitations, and to assess their suitability for different risk management purposes. The remainder of the paper is 
organized as follows: ... 

\section{Top-down vs. Bottom-Up Models}

\noindent Portfolio credit risk models can be broadly divided into bottom-up and top-down approaches, depending on whether credit
risk is modeled at the level of individual borrowers or directly at the level of portfolio segments. This distinction reflects 
two fundamentally different ways of thinking about credit risk aggregation, modeling losses as the outcome of individual 
firm-level default mechanisms, or modeling them as the result of common systematic risk factors affecting groups of exposures simultaneously. \\

\noindent Bottom-up models build portfolio risk starting from the individual instrument or borrower. Each exposure is characterized by 
borrower-specific credit risk parameters such as the probability of default, loss given default, and exposure at default, while 
dependence across borrowers is introduced through correlated risk drivers. In this framework, portfolio losses arise from the joint 
realization of individual default events. As emphasized in the credit risk literature, this approach is closely related to models that 
provide an explicit economic or statistical description of default at the firm level, including structural and intensity-based 
specifications (Lando, 2004). Aggregation across borrowers then delivers the portfolio loss distribution. \\

\noindent A key advantage of bottom-up models is their high degree of granularity. They allow risk managers to measure concentration 
risk, assess marginal risk contributions of individual exposures, and perform detailed scenario analyses at the borrower level. 
This makes them particularly suitable for corporate loan portfolios and portfolios of traded credit instruments, where firm-level 
information is available and economically meaningful. However, the bottom-up approach is also data-intensive and requires strong 
assumptions regarding default dependence, which must be calibrated using limited historical default data or proxy measures such 
as asset correlations. \\

\noindent In contrast, top-down models describe credit risk directly at the level of portfolio segments, such as industries, 
regions, or rating classes. Instead of modeling each borrower separately, these models focus on common sources of risk 
(most notably macroeconomic conditions) that drive default rates and losses across groups of borrowers. From this perspective, 
idiosyncratic risk is largely diversified away, and portfolio losses are primarily determined by the evolution of systematic 
factors. As discussed by Lando (2004), such models are particularly useful when individual default mechanisms are difficult to 
observe or estimate reliably, but aggregate default behavior exhibits a stable relationship with economic conditions. \\

\noindent Top-down models are therefore especially well suited for stress testing and scenario analysis, where the objective 
is to assess the sensitivity of portfolio losses to adverse macroeconomic developments. They are also computationally simpler and 
less demanding in terms of input data, which makes them attractive for large, relatively homogeneous portfolios, such as retail credit 
exposures. Their main limitation lies in the lack of borrower-level detail, which prevents a precise assessment of individual risk 
contributions and limits their usefulness for pricing and active portfolio optimization. \\

\noindent In practice, the distinction between top-down and bottom-up models highlights a fundamental trade-off between economic detail
and tractability. Bottom-up models provide richer insights into individual credit risk and portfolio composition, while top-down models 
offer a clearer link between credit losses and systematic risk factors. For this reason, modern risk management frameworks often combine 
both approaches, applying them selectively depending on the portfolio structure and the specific risk management objective. \\


\end{document}
