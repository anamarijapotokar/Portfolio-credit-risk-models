\documentclass[12pt]{article}

\usepackage[utf8]{inputenc}
\usepackage[english]{babel} 
\usepackage{graphicx}
\usepackage{hyperref}
\usepackage{geometry}
\usepackage{titlesec}
\usepackage{csquotes}
\usepackage{biblatex}
\usepackage{array}
\titlelabel{\thetitle.\quad}
\geometry{a4paper, margin=2.5cm}

\title{Portfolio Credit Risk models}
\author{Anamarija Potokar, Živa Artnak}
\date{}

\begin{document}

\maketitle

\section{Introduction}

Credit risk is the possibility that a borrower will not meet its contractual obligations, leading to a loss for the lender. In practice, this risk is usually described in terms of \textbf{probability of default} and the \textbf{loss given default}. There are two broad ways in which this risk is modelled. In \textbf{default-mode} approaches, only the event of default matters, meaning a borrower either survives the risk horizon or defaults, and losses occur only in the latter case. In \textbf{rating-migration} or \textbf{mark-to-market} approaches, the focus is broader. Changes in a borrower's credit quality, reflected by transitions between rating categories, generate gains and losses through spread changes even if no default occurs. Both perspectives are important, but they lead to different modelling choices and risk measures. \\

\noindent Traditionally, banks assessed credit risk mainly at the level of individual transactions, often with a simple classification of loans into "good" and "bad"; this view is no longer sufficient. Experience has shown that virtually any exposure can become "bad" under adverse economic conditions, and that the risk of a bank cannot be understood by looking at single loans in isolation. What matters is the joint behaviour of many borrowers and the way losses can accumulate in stressed states of the economy. As a result, financial institutions increasingly measure and manage credit risk at the portfolio level, asking not only what is the risk of this loan, but also how does this loan affect the risk of the whole book. \\

\noindent Several developments have reinforced this portfolio perspective. Competitive pressure and narrow lending margins leave little room for error in selecting and pricing individual exposures; small mistakes can have a large impact on profitablity once they are aggregated. At the same time, markets for syndicated loans, securitisations and credit derivatives have become more liquid, giving banks more tools to actively reshape their credit portfolios after origination. Diversification across borrowers, sectors and regions, and the timing of adjustments to the portfolio, increasingly determine whether a bank earns a profit or suffers a loss in a given period. \\

\noindent To take advantage of these opportunities, however, banks must be able to answer a number of technical questions. They need to quantify the distribution of portfolio loss over a given horizon, not only its expected value but also extreme outcomes that occur with low probability. They must understand how different macroeconomic scenarios or sector-specific shocks affect this loss distribution, how changes in the portfolio mix alter concentration and diversification, and how risk-based pricing and capital allocation should reflect expected losses and the cost of holding credit risk capital. \\

\noindent This essay addresses these questions by reviewing and comparing the main portfolio credit risk models used in risk management and regulation. We discuss bottom-up versus top-down approaches, default-mode versus mark-to-market frameworks, and structural versus reduced-form models, and we relate them to the regulatory capital model used in the Basel II/III Internal Ratings-Basel approach. The aim is not to provide a full mathematical treatment of each model, but to explain their assumptions, inputs and outputs, to highlight their strengths and weaknesses, and to draw clear conclusions about their suitability for different risk management purposes. \\

\section{Top-down vs. Bottom-Up models}

\noindent Portfolio credit risk models can be broadly divided into bottom-up and top-down approaches, depending on whether credit
risk is modeled at the level of individual borrowers or directly at the level of portfolio segments. This distinction reflects 
two fundamentally different ways of thinking about credit risk aggregation, modeling losses as the outcome of individual 
firm-level default mechanisms, or modeling them as the result of common systematic risk factors affecting groups of exposures simultaneously. \\

\noindent Bottom-up models build portfolio risk starting from the individual instrument or borrower. Each exposure is characterized by 
borrower-specific credit risk parameters such as the probability of default, loss given default, and exposure at default, while 
dependence across borrowers is introduced through correlated risk drivers. In this framework, portfolio losses arise from the joint 
realization of individual default events. This approach is closely related to models that 
provide an explicit economic or statistical description of default at the firm level, including structural and intensity-based 
specifications. Aggregation across borrowers then delivers the portfolio loss distribution. \\

\noindent A key advantage of bottom-up models is their high degree of granularity. They allow risk managers to measure concentration 
risk, assess marginal risk contributions of individual exposures, and perform detailed scenario analyses at the borrower level. 
This makes them particularly suitable for corporate loan portfolios and portfolios of traded credit instruments, where firm-level 
information is available and economically meaningful. However, the bottom-up approach is also data-intensive and requires strong 
assumptions regarding default dependence, which must be calibrated using limited historical default data or proxy measures such 
as asset correlations. \\

\noindent In contrast, top-down models describe credit risk directly at the level of portfolio segments, such as industries, 
regions, or rating classes. Instead of modeling each borrower separately, these models focus on common sources of risk 
(most notably macroeconomic conditions) that drive default rates and losses across groups of borrowers. From this perspective, 
idiosyncratic risk is largely diversified away, and portfolio losses are primarily determined by the evolution of systematic 
factors. Such models are particularly useful when individual default mechanisms are difficult to 
observe or estimate reliably, but aggregate default behavior exhibits a stable relationship with economic conditions. \\

\noindent Top-down models are therefore especially well suited for stress testing and scenario analysis, where the objective 
is to assess the sensitivity of portfolio losses to adverse macroeconomic developments. They are also computationally simpler and 
less demanding in terms of input data, which makes them attractive for large, relatively homogeneous portfolios, such as retail credit 
exposures. Their main limitation lies in the lack of borrower-level detail, which prevents a precise assessment of individual risk 
contributions and limits their usefulness for pricing and active portfolio optimization. \\

\noindent In practice, the distinction between top-down and bottom-up models highlights a fundamental trade-off between economic detail
and tractability. Bottom-up models provide richer insights into individual credit risk and portfolio composition, while top-down models 
offer a clearer link between credit losses and systematic risk factors. For this reason, modern risk management frameworks often combine 
both approaches, applying them selectively depending on the portfolio structure and the specific risk management objective. \\

\section{Default-mode vs. Mark-to-market models}

\noindent A second fundamental distinction among portfolio credit risk models concerns whether losses are modeled only through default events or 
whether changes in credit quality short of default are also taken into account. This leads to a classification into default-mode and mark-to-market models, 
a distinction emphasized in both academic literature and practical risk management frameworks. \\

\noindent Default-mode models focus exclusively on the occurrence of default over a given risk horizon. In these models, each exposure either survives or defaults. 
Changes in credit quality that do not lead to default are irrelevant for the loss calculation. As a result, default-mode models are primarily concerned with 
estimating the distribution of default losses, rather than the market value of the portfolio. CreditRisk+ is a prominent example of this approach, as it models 
defaults as random events driven by default intensities and derives the portfolio loss distribution using actuarial techniques. \\

\noindent The main strength of default-mode models lies in their conceptual simplicity and computational efficiency. By abstracting from market value fluctuations, 
these models avoid the need to specify credit spreads, rating transition dynamics, or valuation models for credit instruments. This makes them particularly attractive 
for applications such as economic capital calculation, where the primary objective is to quantify extreme losses at high confidence levels. From a theoretical perspective, 
default-mode models are consistent with the view that credit risk materializes primarily through rare but severe events.
However, by construction, default-mode models ignore potentially important sources of risk. In practice, the market value of credit portfolios can fluctuate significantly 
even in the absence of defaults, due to changes in credit spreads, rating migrations, and shifts in investors’ risk perceptions. These effects are particularly relevant 
for portfolios containing traded credit instruments or loans that are marked to market. \\

\noindent Mark-to-market models explicitly account for such changes in credit quality. In addition to default losses, they capture gains and losses arising from rating 
migrations and spread movements, even when no default occurs. In this framework, portfolio credit risk is measured as the distribution of changes in portfolio value over 
the risk horizon. CreditMetrics, developed by J.P. Morgan, is the canonical example of a mark-to-market model. It models rating transitions using empirical transition 
matrices and revalues each exposure under different rating scenarios, including default, thereby producing a full distribution of portfolio value changes. \\

\noindent Mark-to-market models offer a more comprehensive view of credit risk, particularly for actively managed portfolios and portfolios of traded instruments. 
They are well suited for short to medium term risk horizons and for risk measurement frameworks that rely on value-at-risk or similar metrics. At the same time, they require more extensive modeling assumptions, including assumptions about rating transition probabilities, recovery rates, and the valuation of credit instruments under different credit states. \\

\noindent In practice, the choice between default-mode and mark-to-market models depends on the risk management objective and the nature of the portfolio. Default-mode models are typically preferred for long-horizon capital adequacy and regulatory purposes, while mark-to-market models are more appropriate for portfolios where interim valuation changes are economically significant. Many institutions therefore use both approaches in parallel, recognizing that they capture complementary dimensions of portfolio credit risk.

\section{Conditional vs. Unconditional models}

\noindent Another important distinction in portfolio credit risk modeling concerns whether default probabilities (and, more generally, losses) are treated as fixed, long-run quantities or as outcomes that vary with the state of the economy. This leads to the classification into unconditional and conditional models. Although the difference can sound technical, it matters a lot in practice because it determines whether a model produces a relatively stable, ``average'' view of risk or whether it explicitly reacts to changing macro-financial conditions and can be used for scenario-based analysis. \\

\noindent In an unconditional model, key credit risk inputs such as the probability of default and loss given default are typically treated as fixed over the risk horizon. The model aims to produce an overall loss distribution that reflects a long-run average across many possible economic states. Institutions often implement this perspective using so-called through-the-cycle estimates, which are constructed to be relatively stable over time and not overly sensitive to short-run fluctuations in the economy. A practical advantage of this approach is that it supports capital planning and performance measurement in a way that does not change dramatically from quarter to quarter. It also reduces the danger of procyclical decision-making, where risk measures rise sharply in downturns and fall sharply in booms, amplifying the cycle. \\

\noindent Importantly, unconditional models do not ignore dependence across borrowers. They still incorporate the idea that borrowers are exposed to common drivers and that defaults can cluster. The key point is that the dependence structure and default probabilities are typically parameterized in a time-invariant way rather than explicitly linked to observable economic conditions. The model therefore captures average dependence, but it does not directly explain how default risk changes when the macro environment deteriorates. \\

\noindent In contrast, conditional models treat credit risk as state-dependent. They explicitly represent the idea that default rates and losses rise in recessions and fall in expansions. The basic modeling logic is that borrowers are exposed to common systematic conditions (such as broad macroeconomic factors, sector-specific conditions, or financial market stress), and that these conditions influence many borrowers at the same time. Conditional models are therefore especially useful when the goal is not only to measure risk, but to understand why risk changes and how the portfolio might behave under adverse scenarios. \\

\noindent A central concept in many conditional frameworks is conditional independence: borrowers may be treated as largely independent once the systematic environment is fixed, but they become dependent in the unconditional sense because they all respond to the same underlying conditions. This is not just a convenient assumption; it reflects the empirical observation that default clustering is largely driven by shared economic stress. As a result, conditional models provide a natural foundation for stress testing. Rather than simply reporting a single number for risk, they allow a risk manager to translate a macroeconomic scenario into implied probabilities of default, losses given default, default rates, and ultimately portfolio losses. \\

\noindent The conditional versus unconditional distinction also appears in how institutions discuss estimation of probability of default. A common practical split is between point-in-time probabilities of default, which reflect current conditions and tend to respond quickly to changes in the economy, and through-the-cycle probabilities of default, which are smoother and closer to long-run averages. Conditional models align naturally with point-in-time estimation and monitoring, while unconditional models align more closely with through-the-cycle estimates used for stable capital frameworks. Many institutions use both in parallel, since they serve different objectives: point-in-time measures are useful for timely risk signals and pricing, while through-the-cycle measures support longer-term capital and planning.

\section{Structural vs. Reduced-form models}

\noindent A further key distinction concerns the way default is modeled at the level of a single borrower. The two dominant paradigms are structural (firm-value) models and reduced-form (intensity-based) models. Both can be used within bottom-up portfolio frameworks, and both can be combined with systematic factors to generate realistic portfolio dependence. However, they differ fundamentally in their interpretation of what default is and in how they connect credit risk to economic or market data. \\

\noindent Structural models explain default as the outcome of a borrower’s underlying economic condition. In these models, default occurs when the value of the borrower’s assets becomes insufficient relative to its liabilities, so default is tied to the borrower’s balance sheet and asset dynamics. The classical intuition is that equity resembles a residual claim on the firm’s assets, which provides an economically meaningful link between leverage, volatility, and default risk. Structural models are attractive because default is not introduced as an arbitrary event; instead, it arises from an explicit mechanism that reflects firm fundamentals. This makes the framework particularly suitable for corporate credit, where concepts such as firm value, leverage, and debt obligations are central. \\




\noindent Building on these considerations, this report provides a structured overview of the main portfolio credit risk 
models used in risk management and regulation. Following the framework presented in the Risk Management lectures, the models 
are classified along several key dimensions: bottom-up vs. top-down approaches, default-mode vs. mark-to-market models, 
conditional vs. unconditional models of default probability, and structural vs. reduced-form models. The aim of the report 
is to explain the underlying assumptions, inputs, and outputs of these approaches, to highlight their respective strengths 
and limitations, and to assess their suitability for different risk management purposes. The remainder of the paper is 
organized as follows: ... \\

\section{Top-down vs. Bottom-Up models}

\noindent Portfolio credit risk models can be broadly divided into bottom-up and top-down approaches, depending on whether credit
risk is modeled at the level of individual borrowers or directly at the level of portfolio segments. This distinction reflects 
two fundamentally different ways of thinking about credit risk aggregation, modeling losses as the outcome of individual 
firm-level default mechanisms, or modeling them as the result of common systematic risk factors affecting groups of exposures simultaneously. \\

\noindent Bottom-up models build portfolio risk starting from the individual instrument or borrower. Each exposure is characterized by 
borrower-specific credit risk parameters such as the probability of default, loss given default, and exposure at default, while 
dependence across borrowers is introduced through correlated risk drivers. In this framework, portfolio losses arise from the joint 
realization of individual default events. As emphasized in the credit risk literature, this approach is closely related to models that 
provide an explicit economic or statistical description of default at the firm level, including structural and intensity-based 
specifications (Lando, 2004). Aggregation across borrowers then delivers the portfolio loss distribution. \\

\noindent A key advantage of bottom-up models is their high degree of granularity. They allow risk managers to measure concentration 
risk, assess marginal risk contributions of individual exposures, and perform detailed scenario analyses at the borrower level. 
This makes them particularly suitable for corporate loan portfolios and portfolios of traded credit instruments, where firm-level 
information is available and economically meaningful. However, the bottom-up approach is also data-intensive and requires strong 
assumptions regarding default dependence, which must be calibrated using limited historical default data or proxy measures such 
as asset correlations. \\

\noindent In contrast, top-down models describe credit risk directly at the level of portfolio segments, such as industries, 
regions, or rating classes. Instead of modeling each borrower separately, these models focus on common sources of risk 
(most notably macroeconomic conditions) that drive default rates and losses across groups of borrowers. From this perspective, 
idiosyncratic risk is largely diversified away, and portfolio losses are primarily determined by the evolution of systematic 
factors. As discussed by Lando (2004), such models are particularly useful when individual default mechanisms are difficult to 
observe or estimate reliably, but aggregate default behavior exhibits a stable relationship with economic conditions. \\

\noindent Top-down models are therefore especially well suited for stress testing and scenario analysis, where the objective 
is to assess the sensitivity of portfolio losses to adverse macroeconomic developments. They are also computationally simpler and 
less demanding in terms of input data, which makes them attractive for large, relatively homogeneous portfolios, such as retail credit 
exposures. Their main limitation lies in the lack of borrower-level detail, which prevents a precise assessment of individual risk 
contributions and limits their usefulness for pricing and active portfolio optimization. \\

\noindent In practice, the distinction between top-down and bottom-up models highlights a fundamental trade-off between economic detail
and tractability. Bottom-up models provide richer insights into individual credit risk and portfolio composition, while top-down models 
offer a clearer link between credit losses and systematic risk factors. For this reason, modern risk management frameworks often combine 
both approaches, applying them selectively depending on the portfolio structure and the specific risk management objective. \\

\section{Default-mode vs. Mark-to-market models}

\noindent A second fundamental distinction among portfolio credit risk models concerns whether losses are modeled only through default events or 
whether changes in credit quality short of default are also taken into account. This leads to a classification into default-mode and mark-to-market models, 
a distinction emphasized in both academic literature and practical risk management frameworks. \\

\noindent Default-mode models focus exclusively on the occurrence of default over a given risk horizon. In these models, each exposure either survives or defaults. 
Changes in credit quality that do not lead to default are irrelevant for the loss calculation. As a result, default-mode models are primarily concerned with 
estimating the distribution of default losses, rather than the market value of the portfolio. CreditRisk+ is a prominent example of this approach, as it models 
defaults as random events driven by default intensities and derives the portfolio loss distribution using actuarial techniques. \\

\noindent The main strength of default-mode models lies in their conceptual simplicity and computational efficiency. By abstracting from market value fluctuations, 
these models avoid the need to specify credit spreads, rating transition dynamics, or valuation models for credit instruments. This makes them particularly attractive 
for applications such as economic capital calculation, where the primary objective is to quantify extreme losses at high confidence levels. From a theoretical perspective, 
default-mode models are consistent with the view that credit risk materializes primarily through rare but severe events, a perspective discussed extensively in the credit 
risk literature (Lando, 2004).
However, by construction, default-mode models ignore potentially important sources of risk. In practice, the market value of credit portfolios can fluctuate significantly 
even in the absence of defaults, due to changes in credit spreads, rating migrations, and shifts in investors’ risk perceptions. These effects are particularly relevant 
for portfolios containing traded credit instruments or loans that are marked to market. \\

\noindent Mark-to-market models explicitly account for such changes in credit quality. In addition to default losses, they capture gains and losses arising from rating 
migrations and spread movements, even when no default occurs. In this framework, portfolio credit risk is measured as the distribution of changes in portfolio value over 
the risk horizon. CreditMetrics, developed by J.P. Morgan, is the canonical example of a mark-to-market model. It models rating transitions using empirical transition 
matrices and revalues each exposure under different rating scenarios, including default, thereby producing a full distribution of portfolio value changes. \\

\noindent Mark-to-market models offer a more comprehensive view of credit risk, particularly for actively managed portfolios and portfolios of traded instruments. 
They are well suited for short to medium term risk horizons and for risk measurement frameworks that rely on value-at-risk or similar metrics. At the same time, they require more extensive modeling assumptions, including assumptions about rating transition probabilities, recovery rates, and the valuation of credit instruments under different credit states. 
\textbf{As noted in the literature, this increased realism comes at the cost of greater model complexity and sensitivity to parameter estimation (Lando, 2004).} (to k je boldan nevem ce bi dali not?) \\

\noindent In practice, the choice between default-mode and mark-to-market models depends on the risk management objective and the nature of the portfolio. Default-mode models are typically preferred for long-horizon capital adequacy and regulatory purposes, while mark-to-market models are more appropriate for portfolios where interim valuation changes are economically significant. Many institutions therefore use both approaches in parallel, recognizing that they capture complementary dimensions of portfolio credit risk. \\

\end{document}
