\documentclass[12pt]{article}

\usepackage[utf8]{inputenc}
\usepackage[english]{babel} 
\usepackage{graphicx}
\usepackage{hyperref}
\usepackage{geometry}
\usepackage{titlesec}
\usepackage{csquotes}
\usepackage{biblatex}
\usepackage{array}
\titlelabel{\thetitle.\quad}
\geometry{a4paper, margin=2.5cm}

\title{Portfolio Credit Risk models}
\author{Anamarija Potokar, Živa Artnak}
\date{}

\begin{document}

\maketitle

\section{Introduction}

Credit risk is the possibility that a borrower will not meet its contractual obligations, leading to a loss for the lender. In practice, this risk is usually described in terms of \textbf{probability of default} and the \textbf{loss given default}. There are two broad ways in which this risk is modelled. In \textbf{default-mode} approaches, only the event of default matters, meaning a borrower either survives the risk horizon or defaults, and losses occur only in the latter case. In \textbf{rating-migration} or \textbf{mark-to-market} approaches, the focus is broader. Changes in a borrower's credit quality, reflected by transitions between rating categories, generate gains and losses through spread changes even if no default occurs. Both perspectives are important, but they lead to different modelling choices and risk measures. \\

\noindent Traditionally, banks assessed credit risk mainly at the level of individual transactions, often with a simple classification of loans into "good" and "bad"; this view is no longer sufficient. Experience has shown that virtually any exposure can become "bad" under adverse economic conditions, and that the risk of a bank cannot be understood by looking at single loans in isolation. What matters is the joint behaviour of many borrowers and the way losses can accumulate in stressed states of the economy. As a result, financial institutions increasingly measure and manage credit risk at the portfolio level, asking not only what is the risk of this loan, but also how does this loan affect the risk of the whole book. \\

\noindent Several developments have reinforced this portfolio perspective. Competitive pressure and narrow lending margins leave little room for error in selecting and pricing individual exposures; small mistakes can have a large impact on profitablity once they are aggregated. At the same time, markets for syndicated loans, securitisations and credit derivatives have become more liquid, giving banks more tools to actively reshape their credit portfolios after origination. Diversification across borrowers, sectors and regions, and the timing of adjustments to the portfolio, increasingly determine whether a bank earns a profit or suffers a loss in a given period. \\

\noindent To take advantage of these opportunities, however, banks must be able to answer a number of technical questions. They need to quantify the distribution of portfolio loss over a given horizon, not only its expected value but also extreme outcomes that occur with low probability. They must understand how different macroeconomic scenarios or sector-specific shocks affect this loss distribution, how changes in the portfolio mix alter concentration and diversification, and how risk-based pricing and capital allocation should reflect expected losses and the cost of holding credit risk capital. \\

\noindent This essay addresses these questions by reviewing and comparing the main portfolio credit risk models used in risk management and regulation. We discuss bottom-up versus top-down approaches, default-mode versus mark-to-market frameworks, and structural versus reduced-form models, and we relate them to the regulatory capital model used in the Basel II/III Internal Ratings-Based (IRB) approach. The aim is not to provide a full mathematical treatment of each model, but to explain their assumptions, inputs and outputs, to highlight their strengths and weaknesses, and to draw clear conclusions about their suitability for different risk management purposes.

\end{document}
