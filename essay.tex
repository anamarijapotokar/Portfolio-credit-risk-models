\documentclass[12pt]{article}

\usepackage[utf8]{inputenc}
\usepackage[english]{babel} 
\usepackage{graphicx}
\usepackage{hyperref}
\usepackage{geometry}
\usepackage{titlesec}
\usepackage{csquotes}
\usepackage{biblatex}
\usepackage{array}
\titlelabel{\thetitle.\quad}
\geometry{a4paper, margin=2.5cm}

\title{Portfolio Credit Risk models}
\author{Anamarija Potokar, Živa Artnak}
\date{}

\begin{document}

\maketitle

\section{Introduction}

Credit risk is the possibility that a borrower will not meet its contractual obligations, leading to a loss for the lender. In practice, this risk is usually described in terms of \textbf{probability of default} and the \textbf{loss given default}. There are two broad ways in which this risk is modelled. In \textbf{default-mode} approaches, only the event of default matters, meaning a borrower either survives the risk horizon or defaults, and losses occur only in the latter case. In \textbf{rating-migration} or \textbf{mark-to-market} approaches, the focus is broader. Changes in a borrower's credit quality, reflected by transitions between rating categories, generate gains and losses through spread changes even if no default occurs. Both perspectives are important, but they lead to different modelling choices and risk measures. \\

\noindent Traditionally, banks assessed credit risk mainly at the level of individual transactions, often with a simple classification of loans into "good" and "bad"; this view is no longer sufficient. Experience has shown that virtually any exposure can become "bad" under adverse economic conditions, and that the risk of a bank cannot be understood by looking at single loans in isolation. What matters is the joint behaviour of many borrowers and the way losses can accumulate in stressed states of the economy. As a result, financial institutions increasingly measure and manage credit risk at the portfolio level, asking not only what is the risk of this loan, but also how does this loan affect the risk of the whole book. \\

\noindent Several developments have reinforced this portfolio perspective. Competitive pressure and narrow lending margins leave little room for error in selecting and pricing individual exposures; small mistakes can have a large impact on profitablity once they are aggregated. At the same time, markets for syndicated loans, securitisations and credit derivatives have become more liquid, giving banks more tools to actively reshape their credit portfolios after origination. Diversification across borrowers, sectors and regions, and the timing of adjustments to the portfolio, increasingly determine whether a bank earns a profit or suffers a loss in a given period. \\

\noindent To take advantage of these opportunities, however, banks must be able to answer a number of technical questions. They need to quantify the distribution of portfolio loss over a given horizon, not only its expected value but also extreme outcomes that occur with low probability. They must understand how different macroeconomic scenarios or sector-specific shocks affect this loss distribution, how changes in the portfolio mix alter concentration and diversification, and how risk-based pricing and capital allocation should reflect expected losses and the cost of holding credit risk capital. \\

\noindent This essay addresses these questions by reviewing and comparing the main portfolio credit risk models used in risk management and regulation. We discuss bottom-up versus top-down approaches, default-mode versus mark-to-market frameworks, and structural versus reduced-form models, and we relate them to the regulatory capital model used in the Basel II/III Internal Ratings-Basel approach. The aim is not to provide a full mathematical treatment of each model, but to explain their assumptions, inputs and outputs, to highlight their strengths and weaknesses, and to draw clear conclusions about their suitability for different risk management purposes. \\

\noindent Building on these considerations, this report provides a structured overview of the main portfolio credit risk 
models used in risk management and regulation. Following the framework presented in the Risk Management lectures, the models 
are classified along several key dimensions: bottom-up vs. top-down approaches, default-mode vs. mark-to-market models, 
conditional vs. unconditional models of default probability, and structural vs. reduced-form models. The aim of the report 
is to explain the underlying assumptions, inputs, and outputs of these approaches, to highlight their respective strengths 
and limitations, and to assess their suitability for different risk management purposes. The remainder of the paper is 
organized as follows: ... \\

\section{Top-down vs. Bottom-Up models}

\noindent Portfolio credit risk models can be broadly divided into bottom-up and top-down approaches, depending on whether credit
risk is modeled at the level of individual borrowers or directly at the level of portfolio segments. This distinction reflects 
two fundamentally different ways of thinking about credit risk aggregation, modeling losses as the outcome of individual 
firm-level default mechanisms, or modeling them as the result of common systematic risk factors affecting groups of exposures simultaneously. \\

\noindent Bottom-up models build portfolio risk starting from the individual instrument or borrower. Each exposure is characterized by 
borrower-specific credit risk parameters such as the probability of default, loss given default, and exposure at default, while 
dependence across borrowers is introduced through correlated risk drivers. In this framework, portfolio losses arise from the joint 
realization of individual default events. As emphasized in the credit risk literature, this approach is closely related to models that 
provide an explicit economic or statistical description of default at the firm level, including structural and intensity-based 
specifications (Lando, 2004). Aggregation across borrowers then delivers the portfolio loss distribution. \\

\noindent A key advantage of bottom-up models is their high degree of granularity. They allow risk managers to measure concentration 
risk, assess marginal risk contributions of individual exposures, and perform detailed scenario analyses at the borrower level. 
This makes them particularly suitable for corporate loan portfolios and portfolios of traded credit instruments, where firm-level 
information is available and economically meaningful. However, the bottom-up approach is also data-intensive and requires strong 
assumptions regarding default dependence, which must be calibrated using limited historical default data or proxy measures such 
as asset correlations. \\

\noindent In contrast, top-down models describe credit risk directly at the level of portfolio segments, such as industries, 
regions, or rating classes. Instead of modeling each borrower separately, these models focus on common sources of risk 
(most notably macroeconomic conditions) that drive default rates and losses across groups of borrowers. From this perspective, 
idiosyncratic risk is largely diversified away, and portfolio losses are primarily determined by the evolution of systematic 
factors. As discussed by Lando (2004), such models are particularly useful when individual default mechanisms are difficult to 
observe or estimate reliably, but aggregate default behavior exhibits a stable relationship with economic conditions. \\

\noindent Top-down models are therefore especially well suited for stress testing and scenario analysis, where the objective 
is to assess the sensitivity of portfolio losses to adverse macroeconomic developments. They are also computationally simpler and 
less demanding in terms of input data, which makes them attractive for large, relatively homogeneous portfolios, such as retail credit 
exposures. Their main limitation lies in the lack of borrower-level detail, which prevents a precise assessment of individual risk 
contributions and limits their usefulness for pricing and active portfolio optimization. \\

\noindent In practice, the distinction between top-down and bottom-up models highlights a fundamental trade-off between economic detail
and tractability. Bottom-up models provide richer insights into individual credit risk and portfolio composition, while top-down models 
offer a clearer link between credit losses and systematic risk factors. For this reason, modern risk management frameworks often combine 
both approaches, applying them selectively depending on the portfolio structure and the specific risk management objective. \\

\section{Default-mode vs. Mark-to-market models}

\noindent A second fundamental distinction among portfolio credit risk models concerns whether losses are modeled only through default events or 
whether changes in credit quality short of default are also taken into account. This leads to a classification into default-mode and mark-to-market models, 
a distinction emphasized in both academic literature and practical risk management frameworks. \\

\noindent Default-mode models focus exclusively on the occurrence of default over a given risk horizon. In these models, each exposure either survives or defaults. 
Changes in credit quality that do not lead to default are irrelevant for the loss calculation. As a result, default-mode models are primarily concerned with 
estimating the distribution of default losses, rather than the market value of the portfolio. CreditRisk+ is a prominent example of this approach, as it models 
defaults as random events driven by default intensities and derives the portfolio loss distribution using actuarial techniques. \\

\noindent The main strength of default-mode models lies in their conceptual simplicity and computational efficiency. By abstracting from market value fluctuations, 
these models avoid the need to specify credit spreads, rating transition dynamics, or valuation models for credit instruments. This makes them particularly attractive 
for applications such as economic capital calculation, where the primary objective is to quantify extreme losses at high confidence levels. From a theoretical perspective, 
default-mode models are consistent with the view that credit risk materializes primarily through rare but severe events, a perspective discussed extensively in the credit 
risk literature (Lando, 2004).
However, by construction, default-mode models ignore potentially important sources of risk. In practice, the market value of credit portfolios can fluctuate significantly 
even in the absence of defaults, due to changes in credit spreads, rating migrations, and shifts in investors’ risk perceptions. These effects are particularly relevant 
for portfolios containing traded credit instruments or loans that are marked to market. \\

\noindent Mark-to-market models explicitly account for such changes in credit quality. In addition to default losses, they capture gains and losses arising from rating 
migrations and spread movements, even when no default occurs. In this framework, portfolio credit risk is measured as the distribution of changes in portfolio value over 
the risk horizon. CreditMetrics, developed by J.P. Morgan, is the canonical example of a mark-to-market model. It models rating transitions using empirical transition 
matrices and revalues each exposure under different rating scenarios, including default, thereby producing a full distribution of portfolio value changes. \\

\noindent Mark-to-market models offer a more comprehensive view of credit risk, particularly for actively managed portfolios and portfolios of traded instruments. 
They are well suited for short to medium term risk horizons and for risk measurement frameworks that rely on value-at-risk or similar metrics. At the same time, they require more extensive modeling assumptions, including assumptions about rating transition probabilities, recovery rates, and the valuation of credit instruments under different credit states. 
\textbf{As noted in the literature, this increased realism comes at the cost of greater model complexity and sensitivity to parameter estimation (Lando, 2004).} (to k je boldan nevem ce bi dali not?) \\

\noindent In practice, the choice between default-mode and mark-to-market models depends on the risk management objective and the nature of the portfolio. Default-mode models are typically preferred for long-horizon capital adequacy and regulatory purposes, while mark-to-market models are more appropriate for portfolios where interim valuation changes are economically significant. Many institutions therefore use both approaches in parallel, recognizing that they capture complementary dimensions of portfolio credit risk. \\

\section{Conditional vs. Unconditional models of default probability}

\noindent An additional important distinction among portfolio credit risk models concerns whether default probabilities are assumed to be fixed over time or explicitly linked to the state of the economy. This leads to a classification into conditional and unconditional models of default probability. The distinction is central to understanding how portfolio credit risk evolves over the business cycle and how models perform under stressed economic conditions. \\

\noindent Conditional models allow default probabilities to vary systematically with macroeconomic conditions. In these models, default risk is explicitly conditioned on observable economic variables such as GDP growth, interest rates, unemployment, or sector-specific indicators. As a result, default probabilities increase during recessions and decline during economic expansions, generating a more realistic representation of cyclical credit risk. This approach reflects the empirical evidence that defaults are not independent over time but are strongly influenced by common economic factors. \\

\noindent A prominent example of a conditional framework is CreditPortfolioView, which models default probabilities using econometric relationships between defaults and macroeconomic variables, following the approach proposed by Wilson (1998). Portfolio loss distributions are then simulated under different macroeconomic scenarios, making such models particularly well suited for stress testing and scenario analysis. The importance of this perspective is also emphasized in the financial stability literature, which highlights that systemic credit risk materializes through simultaneous increases in default probabilities across sectors during adverse economic conditions. \\

\noindent Despite their conceptual appeal, conditional models face several practical challenges. Reliable estimation requires long time series of both default data and macroeconomic variables, and the results can be sensitive to model specification and parameter uncertainty. Moreover, macroeconomic relationships estimated in tranquil periods may break down during severe crises. For these reasons, conditional models are often used in conjunction with unconditional approaches, rather than as a complete replacement. \\

\noindent In contrast, unconditional models assume that default probabilities are constant over the risk horizon and independent of macroeconomic conditions. These probabilities are typically estimated from long-run historical averages or credit rating transition matrices. Once calibrated, they are treated as exogenous inputs to the model and remain unchanged regardless of the prevailing economic environment. Many traditional portfolio credit risk models adopt this approach, implicitly assuming that the effects of economic fluctuations are already embedded in the historical default data. \\

\noindent The main advantage of unconditional models lies in their simplicity and stability. They are relatively easy to calibrate and implement, and they avoid the need to specify explicit relationships between default behavior and macroeconomic variables. For this reason, unconditional default probabilities are commonly used in regulatory capital frameworks and long-horizon risk measurement exercises. However, as emphasized in both academic research and policy-oriented studies, this approach has important limitations. In particular, unconditional models tend to underestimate risk during economic expansions and overestimate diversification benefits, while failing to capture the sharp increase in default clustering observed during downturns (Lando, 2004). \\

\noindent In practice, the choice between conditional and unconditional default probabilities reflects a trade-off between stability and realism. Unconditional models provide a 
simple and robust benchmark, while conditional models offer a richer and more responsive representation of portfolio credit risk over the business cycle. Modern risk management 
frameworks increasingly rely on conditional models for stress testing and systemic risk assessment, while retaining unconditional probabilities for regulatory capital and long-term planning purposes. \\

\section{Structural vs. Reduced-form models}

\noindent A final fundamental distinction in portfolio credit risk modeling concerns whether default is modeled as an endogenous economic event or as an exogenous random occurrence. This leads to the classification of models into structural and reduced-form approaches, a distinction that plays a central role in both theoretical and practical credit risk analysis. \\

\noindent Structural models derive default from the economic condition of the firm. Default occurs when the value of a firm’s assets falls below a critical threshold related to its debt obligations. As emphasized in the Risk Management lectures, this approach views default as a predictable outcome of deteriorating fundamentals rather than a sudden surprise. A standard example of this approach is the Merton model, in which equity is interpreted as a call option on the firm’s assets and default occurs when the asset value is insufficient to repay debt at maturity. Extensions of this framework allow for more realistic features, such as early default, stochastic interest rates, and time-varying leverage. \\

\noindent The main strength of structural models lies in their strong economic intuition. They provide a clear link between default risk, capital structure, asset volatility, and leverage, and they naturally generate default dependence through correlated asset values. This makes them particularly attractive for bottom-up portfolio models, such as Moody’s KMV, which use firm-level balance sheet and equity market information to estimate default probabilities. However, structural models also face important limitations. Asset values are not directly observable, calibration can be complex, and empirical credit spreads are often much higher than those implied by the models, suggesting that additional sources of risk are not fully captured (Lando, 2004). \\

\noindent Reduced-form models, in contrast, do not attempt to model the firm’s economic structure explicitly. Instead, default is treated as an exogenous event governed by a stochastic default intensity or hazard rate. Default arrives unexpectedly, and its probability is specified directly as a function of time or observable risk factors. This approach is often described as correlation-based rather than causation-based, as default probabilities are linked to external signals rather than derived from firm fundamentals. \\

\noindent Reduced-form models offer greater flexibility and tractability, particularly for pricing and portfolio applications. They are well suited for modeling time-varying default risk, incorporating macroeconomic factors, and handling large portfolios with limited firm-level information. As discussed in the literature, intensity-based models provide a natural framework for capturing default clustering and for integrating credit risk with term-structure modeling (Lando, 2004). Many default-mode portfolio models, such as CreditRisk+, are rooted in the reduced-form tradition. \\

\noindent In practice, structural and reduced-form models should be viewed as complementary rather than competing approaches. Structural models offer valuable economic insights and are useful for understanding the drivers of default risk at the firm level, while reduced-form models provide practical tools for portfolio risk measurement, stress testing, and regulatory applications. Modern portfolio credit risk frameworks often combine elements of both approaches, balancing economic interpretability with empirical performance and computational efficiency. \\



\end{document}
